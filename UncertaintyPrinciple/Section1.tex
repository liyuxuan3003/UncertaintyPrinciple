\section{物理部分的准备}

\subsection{德布罗意关系式}

\begin{frame}
    \frametitle{德布罗意关系式}
    在大学物理3中,我们曾学过德布罗意关系式,比较熟悉的形式是
    \begin{equation}
        E=h\nu\qquad
        p=\frac{h}{\lambda}\label{德布罗意关系式原始1}
    \end{equation}
    然而这个式子并不对称,有失物理学的美感。
\end{frame}

\begin{frame}
    \frametitle{波的描述}
    在大学物理1中,我们曾学过一维简谐平面机械波的波函数,常用的一种形式是
    \begin{equation}
        f(x,t)=A\cos\qty[2\pi\qty(\nu t-\frac{x}{\lambda})+\phi]\label{一维机械波}
    \end{equation}
    \begin{itemize}
        \item 关于波的时间周期特性的描述,使用的物理量是频率$\nu$\vspace{1ex}
        \item 关于波的空间周期特性的描述,使用的物理量是波长$\lambda$\vspace{1ex}
    \end{itemize}
    很明显,频率$\nu$和波长的$\lambda$的地位是不对等的。
\end{frame}

\begin{frame}
    \frametitle{波的描述}
    \begin{columns}[t] 
        \begin{column}{6cm} 
            波的时间周期性的描述
            \begin{itemize}
                \item 周期$T$\quad 同一空间点上,相隔多长时间振动状态重复一次。
                \item 频率$\nu=1/T$\\单位时间振动状态的重复次数。
                \item 角频率$\omega=2\pi/T=2\pi\nu$\\单位时间的相位变化。
            \end{itemize}
        \end{column}
        \begin{column}{6cm} 
            波的空间周期性的描述
            \begin{itemize}
                \item 波长$\lambda$\quad 同一时间点上,相隔多少距离振动状态重复一次。
                \only<-1>{\item ?\\\vphantom{?}}
                \only<-1>{\item ?\\\vphantom{?}}
                \only<2->{\item 波数$\kappa=1/\lambda$\\单位空间振动状态的重复次数。}
                \only<2->{\item 角波数$k=2\pi/\lambda=2\pi\kappa$\\单位空间的相位变化。}
            \end{itemize}
        \end{column}
    \end{columns}\vspace{3ex}

    \uncover<2->{\centering 形象的说,波数就是单位空间距离上波形的个数。}
\end{frame}

\begin{frame}
    \frametitle{波的描述}
    基于角频率$\omega$和角波数$k$,一维平面简谐波的波函数\eqref{一维机械波}可以重表述为
    \begin{equation}
        f(x,t)=A\cos[\omega t-kx+\phi]\label{一维机械波简化}
    \end{equation}

    引入角波矢$\vb*{k}$,其值为角波数$k$,其方向为波的传播方向,将\eqref{一维机械波简化}由一维推广至三维
    \begin{equation}
        f(\vb*{r},t)=A\cos[\omega t-\vb*{k}\cdot\vb*{r}+\phi]\label{三维机械波简化}
    \end{equation}
\end{frame}

\begin{frame}
    \frametitle{约化普朗克常数}
    现在回到德布罗意关系式\eqref{德布罗意关系式原始1},代入角频率$\omega$和角波矢$\vb*{k}$
    \begin{equation}
        E=\frac{h}{2\pi}\omega\qquad
        \vb*{p}=\frac{h}{2\pi}\vb*{k}\label{德布罗意关系式原始2}
    \end{equation}
    \begin{definition}[约化普朗克常数]
        定义约化普朗克常数$\hbar$为
        \begin{equation}
            \hbar=\frac{h}{2\pi}
        \end{equation}
    \end{definition}
\end{frame}

\begin{frame}
    \frametitle{德布罗意关系式}
    \begin{theorem}[德布罗意关系式]
        德布罗意关系式建立了量子力学中,波和粒子间的关系
        \begin{equation}
            E=\hbar\omega\qquad
            \vb*{p}=\hbar\vb*{k}
        \end{equation}
        即,能量正比于角频率,动量正比于角波矢,比例系数为约化普朗克常数。
    \end{theorem}
\end{frame}

\subsection{波粒二象性}

\begin{frame}
    \frametitle{波粒二象性}
    \begin{center}
        如何理解波粒二象性?
    \end{center}
\end{frame}

\begin{frame}
    \frametitle{片面夸大波动性的观点}
    \begin{itemize}
        \item 观点的内容:电子是一种物质波包。
        \item 观点的缺陷:电子具有原子性,未曾观察到“电子的碎片”。
    \end{itemize}
\end{frame}

\begin{frame}
    \frametitle{片面夸大粒子性的观点}
    \begin{itemize}
        \item 观点的内容:电子波是由大量电子构成的疏密波(类比声波)。
        \item 观点的缺陷:电子的波动性不依赖于大量电子的存在,单个电子也具有波动性!
    \end{itemize}
\end{frame}

\begin{frame}
    \frametitle{电子波的双缝干涉实验}
    电子的波动性,可以仿照光的双缝干涉实验或单缝衍射实验进行验证。
    \begin{columns}[t] 
        \begin{column}{6cm} 
            \begin{figure}
                \centering\includegraphics[width=4.75cm]{build/Section1_01.fig.pdf}
                \caption{双缝干涉实验的器具}
            \end{figure}
        \end{column}
        \begin{column}{6cm} 
            \begin{figure}
                \centering\includegraphics[width=5cm]{build/Section1_02.fig.pdf}
                \caption{双缝干涉图样}
            \end{figure}
            \begin{figure}
                \centering\includegraphics[width=5cm]{build/Section1_03.fig.pdf}
                \caption{单缝衍射图样}
            \end{figure}
        \end{column}
    \end{columns}
\end{frame}

\begin{frame}
    \frametitle{电子波的双缝干涉实验}
    \begin{enumerate}
        \item 令大量电子通过双缝,出现干涉条纹。
        \item 令电子一个一个通过双缝,最初,出现随机的点,一段时间后,出现干涉条纹。
    \end{enumerate}
\end{frame}

\begin{frame}
    \frametitle{电子波的双缝干涉实验}
    关于单电子的双缝干涉实验,常总结为
    \begin{center}
        少量电子\textbf{体现}粒子性,大量电子\textbf{体现}波动性。
    \end{center}
    这个说法很讨巧,但回避了本质,且常被误读为
    \begin{center}
        少量电子\textbf{具有}粒子性,大量电子\textbf{具有}波动性。
    \end{center}
    这个实验恰恰表明了:\textbf{即便是单个电子,也具有波动性!}
\end{frame}

\begin{frame}
    \frametitle{电子波的双缝干涉实验}
    \begin{center}
        电子的波动性不依赖于大量电子的存在,单个电子也具有波动性!
    \end{center}
\end{frame}

\begin{frame}
    \frametitle{粒子?波?}
    经典物理中,粒子应当具有以下两点特性
    \begin{enumerate}
        \item 粒子具有原子性,是不可分割的。
        \item 粒子具有确定的轨道,是可以确定任意时刻粒子的位置和速度。
    \end{enumerate}
    经典物理中,波应当具有以下两点特征
    \begin{enumerate}
        \item 波代表了某种物理量在空间上的分布。
        \item 波具有衍射和干涉的现象。
    \end{enumerate}
\end{frame}

\begin{frame}
    \frametitle{粒子?波?}
    关于量子力学中的波粒二象性,我们到底知道什么?
    \begin{itemize}
        \item 电子确实具有原子性,但其是否具有确定的轨道是未曾验证的。
        \item 电子波可以发生干涉和衍射,但其代表了何种物理量的分布也是未曾确定的。
    \end{itemize}
\end{frame}

\begin{frame}
    \frametitle{波粒二象性}
    \begin{center}
        量子力学中的波粒二象性,只包含了部分的的波动性和部分的粒子性。
    \end{center}
\end{frame}

\subsection{波函数}

\begin{frame}
    \frametitle{单电子干涉实验的统计性}
    单电子干涉实验的结果,表现的其实是电子的统计行为
    \begin{itemize}
        \item 通过的电子数较少时(抛硬币的次数较少时),无显著规律。
        \item 通过的电子数较多时(抛硬币的次数较多),服从干涉图样的确定分布。
    \end{itemize}
\end{frame}

\begin{frame}
    \frametitle{单电子干涉实验的统计性}
    \begin{center}
        量子力学中,粒子的运动是概率的,波动着的是概率。
    \end{center}
\end{frame}

\begin{frame}
    \frametitle{概率?概率密度?}
    \begin{center}
        概率--概率密度\qquad
        质量--质量密度
    \end{center}
\end{frame}

\begin{frame}
    \frametitle{波函数}
    通常用$\psi(\vb*{r},t)$表示波函数,由于概率密度分是布与光强分布一致
    \begin{equation}
        \text{波的强度}\propto|\text{波幅}|^2
    \end{equation}
    因此
    \begin{itemize}
        \item 波函数模的平方$\abs{\psi(\vb*{r},t)}^2$才表示概率密度分布。
        \item 波函数$\psi(\vb*{r},t)$无明确物理意义,通常称为概率密度幅。
    \end{itemize}
    以上,即是波函数的统计诠释。
\end{frame}